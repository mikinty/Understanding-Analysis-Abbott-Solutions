%% 7.2 %%

\setcounter{section}{1}
\section{The Definition of the Riemann Integral}
\setcounter{exercise}{0}

The partition theorems may be a little confusing at first to understand.
The basic idea is that if we have a refinement of some partition $P$,
then we have all the original points of $P$, with additional points 
in between.

Now, if we consider the $L(f, P)$, we see that adding more points 
in some interval $[x_k, x_{k+1}]$ 
will only yield estimates $m_k$ that are larger than the $\inf(f(x), x \in [x_k, x_{k+1}])$,
since in this interval, any $f(x)$ will be larger than the $\inf$.
Therefore, we can conclude 
\begin{equation*}
  L(f, P) \leq L(f, \text{refinement})
\end{equation*}
On the other hand, if we look at $U(f, P)$, we get the opposite relation.
The refinement produces a $U$ estimate that is less than the original, 
since if you have some maximum over some interval, adding more points 
gives you chance to take estimates that are less than the maximum.
Therefore, we conclude 
\begin{equation*}
  U(f, P) \geq U(f, \text{refinement})
\end{equation*}

There is also a typo in Lemma 7.2.4 -- the author meant to say 
if $Q = P_1 \cup P_2$, then $P_1 \subseteq Q, P_2, \subseteq Q$.

\bx{

}