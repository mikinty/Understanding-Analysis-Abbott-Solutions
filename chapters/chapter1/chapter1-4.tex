%%% 1.4 %%%
\section{Consequences of Completeness}
\setcounter{exercise}{0}

\bx{
If $a<0$, then we have two cases,
\begin{enumerate}
	\item If $b > 0$, then $a < 0 < b$.
	\item If $b = 0$, then we can take $-b, -a$, which satisfies $0 \leq -b < -a$, and apply Theorem 1.4.3.
\end{enumerate}
}


\bx{
\ea{
\item If $a, b \in \mathbb{Q}$, then
\begin{align*}
	a &= \frac{a_1}{a_2} \\
	b &= \frac{b_1}{b_2} \\
	\implies a + b &= \frac{a_1b_2 + a_2b_1}{a_2b_2} \in \mathbb{Q}
\end{align*}
\item We can use contradiction,
\begin{itemize}
	\item \AFSOC $a + t \in \mathbb{Q}$. Let $a+t = \frac{m}{n}$.
	We know $a = \frac{a_1}{a_2}$ since $a \in \mathbb{Q}$, so
	\begin{align*}
		a + t &= \frac{m}{n}\\
		t &= \frac{m}{n} - \frac{a_1}{a_2} \in \mathbb{Q},
	\end{align*}
	which is a contradiction since we are given $t \in \mathbb{Q}$.
	Therefore, we conclude $a + t \in \mathbb{I}$.

	\item \AFSOC $at \in \mathbb{Q}$.
	Let $at = \frac{m}{n}$.
	We know $a = \frac{a_1}{a_2}$ since $a \in \mathbb{Q}$, so
	\begin{align*}
		at &= \frac{m}{n}\\
		t &= \frac{m}{n}\cdot\frac{a_2}{a_1} \in \mathbb{Q},
	\end{align*}
	which is a contradiction since we are given $t \in \mathbb{Q}$.
	Therefore, we conclude $at \in\mathbb{I}$
\end{itemize}

\item $\mathbb{I}$ is not closed under addition or multiplication.
\begin{align*}
	\pa{3 - \sqrt{2}} + \pa{3 + \sqrt{2}} &= 6 \not\in \mathbb{I}\\
	\pa{3 - \sqrt{2}} \cdot \pa{3 + \sqrt{2}} &= 5 \not\in \mathbb{I}
\end{align*}
}
}

\bx{
	We can apply Theorem 1.4.3, to find $a < q < b, q \in \mathbb{Q}$,
	and then subtract an irrational number such as $\sqrt{2}$ to end up at
	\begin{equation}
		a - \sqrt{2} < q - \sqrt{2} < b - \sqrt{2},
	\end{equation}
	where $q- \sqrt{2} \in \mathbb{I}$.
}

\bx{
Suppose $\exists b$ lower bound such that $b > 0$.
Then by Archimedean Property of $\mathbb{R}$, $\exists n \in \mathbb{N}$
such that $\frac{1}{n} < b$, which means $b$ is not a valid lower bound.
Thus $b \leq 0$, and 0 is a valid lower bound so the $\inf$ is 0.
}

\bx{
\AFSOC $\exists \alpha \in \bigcap_{n=1}^\infty (0, \frac{1}{n})$.
Then $\alpha >0$, but by Archimedean property of reals,
we have that $\exists n \in \mathbb{N} \mid \frac{1}{n} < \alpha$.
Since $\alpha \not\in \pa{0, \frac{1}{n}}$ leads to
$\alpha \not\in \bigcap_{n=1}^\infty (0, \frac{1}{n})$,
a contradiction, we conclude the set is empty.
}

\bx{
\ea{
\item If $\alpha^2 > 2$, then
\begin{align*}
	\pa{a - \frac{1}{n}}^2 &= \alpha^2 - \frac{2\alpha}{n} + \frac{1}{n^2} \\
	&> \alpha^2 - \frac{2\alpha}{n}
\end{align*}
choose $\frac{1}{n_0} < \frac{\alpha^2 - 2}{2\alpha}$. Then
\begin{align*}
	\pa{a - \frac{1}{n_0}}^2 &> \alpha^2 - \frac{2\alpha}{2\alpha}(\alpha^2 - 2)\\
	&> 2
\end{align*}
but $\alpha - \frac{1}{n_0} < \alpha$, so $\alpha$ is not
the least upper bound for the se, so $\alpha \neq \sup T$.
\item Just replace $\sqrt{2}$ with $\sqrt{b}$ for the proof above.
}
}

\bx{
Once we have assigned $g(i) = f(n_i)$, remove $f(n_i)$ from $A$.
Now, there is a new $n_{i+1} = \min\{n \in \mathbb{N}:f(n) \in A\setminus \{f(1), f(2), \dots, f(n_i)\}\}$.
Assign $g(i+1) = f(n_{i+1})$, and repeat.
}

\bx{
\ea{
\item If both are finite, then their union is finite and trivially countable. If one is finite, then first enumerate elements of the finite set. Then map the rest of $\mathbb{N}$ to the countably infinite set. If both are countably infinite, map one set to odds and the other to evens.
\item Induction only holds for finite integers, not infinity.
\item We can arrange each $A_n$ into row $n$ of a $\mathbb{N}\times \mathbb{N}$ matrix. Then, we enumerate by diagonalization.
}
}

\bx{
\ea{
\item If $A \sim B$, then there is a 1-to-1 mapping. We can just take the inverse of the mapping to derive $B \sim A$.
\item If we have $f: A \rightarrow B$, $g: B \rightarrow C$, then we can compose the functions so $g(f(x)): A \rightarrow C$.
}
}

\bx{
The set of all finite subsets of $\mathbb{N}$ can be ordered in increasing order by the sum of each subset.
}

\bx{
\ea{
	\item $f(x) = (x, 0.5) \in S$
	\item Interweave the decimal expansion of $x, y$, e.g.
	\begin{equation}
		f(x, y) = 0.x_1y_1x_2y_2x_3y_3\dots
	\end{equation}
}
}

\bx{
\ea{
\item \begin{align*}
	\sqrt{2}&:x^2 -2 = 0 \\
	\sqrt[3]{2}&:x^3 - 2 = 0
\end{align*}
$\sqrt{3} + \sqrt{2}$ is not as trivial, so we will do it out in more steps.

There are two approaches to finding the integer coefficient polynomial. One is to take advantage of symmetry, and derive that
\begin{equation}
\prod (x - (\pm \sqrt{3} \pm \sqrt{2})
\label{eq:chap1_sym}
\end{equation}
will work (using loose notation of course). A more general technique is to notice that
\begin{align*}
	x &= \sqrt{3} + \sqrt{2} \\
	x^2 &= 5 + 2\sqrt{6} \\
	(x^2 - 5)^2 &= 24 \\
	x^4 - 10x^2 + 1 &= 0.
\end{align*}
Notice that this is actually the exact same answer we get in (\ref{eq:chap1_sym}) if you work it out.
\item Each element of $A_n$ is a root of a $n$ degree polynomial, which we can represent
as an $(n+1)$-tuple of coefficients $\in \mathbb{Z}$. Therefore, $\abs{A_n} = k\abs{\mathbb{N}^{n+1}} = \abs{\mathbb{N}^{n+1}}$,
which we know is countable.
\item We proved earlier in Theorem 1.4.13 that a countably infinite union of countable sets is countable. Since there are a countable number of algebraic numbers, and reals are uncountable, we conclude that transcendentals are also uncountable.
}
}

\bx{
We are proving the \textbf{Schro\"{o}der-Bernstein Theorem}, which states if
there exist 1-to-1 functions $f:X\rightarrow Y$ and $g: Y \rightarrow X$, then
there exists a 1-to-1, onto function $h: X \rightarrow Y$, which implies $X \sim Y$.
\ea{
\item By the definition of 1-to-1, there must be a unique $x \in X$ such that $f(x) = y$.
A 1-to-1 function maps distinct elements from the domain to distinct elements of the range, so if we
take the inverse $f^{-1}$, it will still be 1-to-1, this time from $Y\rightarrow X$.
\label{problem_1_4_13:parta}

\item Possibilities:
\begin{itemize}
	\item Zero: $g^{-1}$ is not guaranteed to be onto, it may not have an inverse for $x$.
	\item Finite: $g^{-1}(x)$ could exist, and similarly for $f^{-1}$. Once the element doesn't exist in the inverse domain,
	the chain will stop.
	\item Infinite: $x$ is in the range of $g$ and the domain of $f$
\end{itemize}
\label{problem_1_4_13:partb}

\item We have 2 cases
\begin{itemize}
	\item The chains are disjoint. Nothing to prove here.
	\item The chains are not disjoint, i.e. they have one common element. Let us call this element $x$.
	We know to the right of $x$, all the elements in the two chains will be equal.
	From the left of $x$, the elements must be equal as well.
	This is because the inverse chain must be unique starting from $x$, see part \ref{problem_1_4_13:parta}.
	Since all the elements are the same, the chains must be the same as well.
\end{itemize}

\item Since we know this chain started with $x\in X$, this $y$ could not have been created from the RHS,
otherwise this $y$ would be in the range of $f$.
Therefore, this chain either has infinite or a finite of elements to the right.
\begin{itemize}
	\item Finite: the chain must start with an element $y \not\in Y$ but not in $f$'s range.
	This is because if we start with $x \in X$, then as mentioned before, all elements $y' \in Y$ will
	be in $f$'s range.
	Therefore, if we start with $y\in Y$, it will match the form indicated.
	\item  Infinite: The chain could not have an infinite elements to the left, because then
	every $y$ must have come from an $f(x')$ for some $x' \in X$.
\end{itemize}
Therefore, these chains only can have a finite number of elements to the left, and it matches the
form indicated.

\item By the definition of $C_x$, all the elements of $y \in C_x$ that are $\in Y$
are mapped by $f$ from $x \in X_1$. This means $f$ maps $X_1$ onto $Y_1$. Similar logic can
be used for $g$ mapping $Y_2$ onto $X_2$.

Since we know $f$ is a 1-to-1 function from $X$ to $Y$, and we just showed it maps $X_1$ to $Y_1$.
We can conclude that $X_1 \sim Y_1$, since $f$ is a bijection between $X_1$ and $Y_1$.
We can similarly conclude $X_2 \sim Y_2$ with $g$.
Since $X_1, X_2$ are a partition of $X$, since all the chains are disjoint, and similarly for $Y_1, Y_2$ of $Y$, and there exists a bijection
$f, g$ for $X_1 \sim Y_1$ and $X_2 \sim Y_2$ respectively, we conclude there must be a bijection between $X$ and $Y$.
Therefore, we conclude $X \sim Y$.

\begin{figure}[H]
	\centering
	\def\ellipseX{1}
	\def\ellipseY{2}
	\begin{tikzpicture}
		\draw (0, 0) ellipse ({\ellipseX} and {\ellipseY});
		\draw (\ellipseX+2, 0) ellipse ({\ellipseX} and {\ellipseY});
		\draw (0, \ellipseY) node[above] {$X$};
		\draw (\ellipseX+2, \ellipseY) node[above] {$Y$};
		\draw (-\ellipseX, 0) -- (\ellipseX, 0);
		\draw (2, 0) -- (2*\ellipseX+2, 0);
		\draw[<->]
			(0, \ellipseY/2) node[below] {$X_1$} --
			(\ellipseX+2, \ellipseY/2) node[below] {$Y_1$}
			node[midway, above] {$f$}
		;
		\draw[<->]
			(0, -\ellipseY/2) node[above] {$X_2$} --
			(\ellipseX+2, -\ellipseY/2) node[above] {$Y_2$}
			node[midway, above] {$g$}
		;
	\end{tikzpicture}
	\caption{$f$ and $g$ mapping $X$ and $Y$}
	\label{fig:problem_1_4_13_bijection}
\end{figure}
}
}
