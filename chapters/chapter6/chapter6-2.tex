%% 6.2 %%

\setcounter{section}{1}
\section{Uniform Convergence of a Sequence of Functions}
\setcounter{exercise}{0}

\bx{
\ea{
\item We take $\lim{n \to \infty}$,
\begin{equation*}
  \lim_{n \to \infty} \frac{
    nx
  }{
    1 + nx^2
  } = \frac{x}{x^2} = \frac{1}{x}
\end{equation*}

\item We can see for 
\begin{equation*}
  \abs{
    \frac{nx}{1 + nx^2} -
    \frac{1}{x}
  } = \frac{
    1
  }{
    x + nx^3
  }
\end{equation*}
which requires us to choose $N$ as
\begin{equation*}
  N > \frac{1 - \epsilon x}{\epsilon x ^3}.
\end{equation*}
this quantity grows as $x \rightarrow 0$, 
so this is not converging uniformly.

\item No, since $x$ can tend towards 0, this will still not converge.

\item Yes, since we can bound $x > 1$, and choose 
\begin{equation*}
  N > \max\pa{\frac{1}{\epsilon} - 1, 1} > 
  \frac{1 - \epsilon x}{\epsilon x ^3}.
\end{equation*}
}
}

\bx{
Pointwise limit is $x/2$, since $\abs{\sin(nx)} \leq 1$.

It is uniformly convergent on all of $\mathbb{R}$, since 
\begin{equation*}
  \abs{
    \frac{
      nx + \sin(nx)
    }{
      2n
    } - 
    \frac{x}{2}
  } = \abs{
    \frac{
      \sin(nx)
    }{
      2n
    }
  } \leq \frac{1}{2n}
\end{equation*}
so we can choose $N > 1/(2\epsilon)$.
}

\bx{
\ea{
\item Pointwise limit should be split into a few cases.
The idea is that $x^n \to 0$ if $x \in [0, 1)$, but $x^n \to \infty$ if $x \in (1, \infty)$.
So we have 
\begin{equation*}
  \lim_{n\to \infty} 
  h_n(x) = \begin{cases}
    x, &x<1\\
    1/2, &x=1\\
    0, &x>1
  \end{cases}
\end{equation*}

\item The limit is discontinuous, at $x=1$, $x \to 1/2 \to 0$.
\item We can show that this sequence converges on $(1, 2)$,
\begin{align*}
  \abs{
    \frac{x}{1+x^n} - 0
  } &< \epsilon\\
  x &< \epsilon + x^n \epsilon\\
  x-\epsilon &< x^n\epsilon\\
  \ln(x-\epsilon) &< n\ln(x) + \ln(\epsilon)\\
  \ln(\frac{x}{\epsilon} - 1) &< n\ln(x)\\
  \frac{
    \ln\pa{\frac{x}{\epsilon} - 1}
  }{
    \ln(x)
  } &\leq \frac{
    \ln\pa{\frac{2}{\epsilon} - 1}
  }{
    \ln(2)
  } < n
\end{align*}
}
}

\bx{
$f_n(x)$ achieves optima points at 
\begin{equation*}
  f'_n(x) = \frac{
    1-nx^2
  }{
    \pa{1+nx^2}^2
  } = 0
\end{equation*}
so when $x = \pm \sqrt{\frac{1}{n}}$. At these points, $f_n(x) = \pm \frac{1}{2\sqrt{n}}$.

The limit function is 0, since the denominator goes to $\infty$.
We can then prove convergence by saying $\abs{f_n(x) - 0} \leq \frac{1}{2\sqrt{n}}$,
and then just choose $n > \frac{1}{4\epsilon^2}$.
\label{chap6:fnx_optima}
}

\bx{
\ea{
\item \begin{equation*}
  \lim_{n\to\infty} f_n(x) = \begin{cases}
    1, &\abs{x} \geq 0\\
    0, &\abs{x} < 0
  \end{cases}
\end{equation*}
This is not uniform convergence since at $x = 0$, we have a discontinuity.
\label{chap6:disc_fn}

\item The idea is that our function from part 
\ref{chap6:disc_fn}
is continuous everywhere, but has freedom to change the convergence
point at $x = 1/n$. If we change this point to converge to $n$ instead
of $1$, so make 
\begin{equation*}
  f_n(x) = \begin{cases}
    1/x, &\abs{x} \geq 1/n\\
    n^2\abs{x}, &\abs{x} < 1/n
  \end{cases}
\end{equation*}
we have that the limit function goes to infinite at $x = 1/n$.
\begin{figure}[H]
  \centering
  \def\tickWidth{2}
  \begin{tikzpicture}
    \draw 
      (-1, 0) -- (3, 0)
      (0, 0) -- (0, 2)
      (0.5, \tickWidth pt) -- 
      (0.5, -\tickWidth pt) node[below]{$\frac{1}{3}$}
      (1, \tickWidth pt) -- 
      (1, -\tickWidth pt) node[below]{$\frac{1}{2}$}
      (2, \tickWidth pt) -- 
      (2, -\tickWidth pt) node[below]{$\frac{1}{1}$}
    ;

    \draw plot [smooth, tension=1] 
      coordinates { 
        (-1,0)
        (0.5, 0.5) 
        (1, 1.5)
      }
    ;
    \draw plot [smooth, tension=1] 
      coordinates { 
        (3, 0)
        (1.5, 0.5) 
        (1, 1.5)
      }
    ;

    \draw plot [smooth, tension=1] 
      coordinates { 
        (-1,0)
        (1.3, 0.5) 
        (2, 1)
      }
    ;
    \draw plot [smooth, tension=1] 
      coordinates { 
        (3, 0)
        (2.3, 0.3) 
        (2, 1)
      }
    ;

    \draw plot [smooth, tension=1] 
      coordinates { 
        (-1,0)
        (0, 0.5) 
        (0.5, 2)
      }
    ;
    \draw plot [smooth, tension=1] 
      coordinates { 
        (3, 0)
        (1, 1) 
        (0.5, 2)
      }
    ;
  \end{tikzpicture}
  \caption{The convergence at $x = 1/n$ rises}
  \label{chap6:fig:converge_1_n}
\end{figure}
}
}

\bx{
($\Rightarrow$)
If we know $(f_n)$ converges uniformly, choose $N$
such that $\abs{f_n(x) - f(x)} < \epsilon/2$,
then we have for this $N$ that if we choose $n, m > N$,
\begin{equation*}
  \abs{f_n(x) - f_m(x)} =
  \abs{
    f_n(x) - f(x) - (f_m(x) - f(x))
  } \leq 
  \abs{f_n(x) - f(x)} + \abs{f_m(x) - f(x)}
  < \frac{\epsilon}{2} +
  \frac{\epsilon}{2} = \epsilon
\end{equation*}

($\Leftarrow$)
We know for some $N$ that for $m, n \geq N$, that 
\begin{equation*}
  \abs{f_n(x) - f_m(x)} < \epsilon \Rightarrow 
  -\epsilon < f_n(x) - f_m(x) < \epsilon
\end{equation*}
Now, we are going to claim that $f(x) = \lim_{n\to \infty} f_n(x)$.

Using the Algebraic Limit Theorem, we can say 
\begin{align*}
  \lim_{n\to\infty} -\epsilon &< 
  \lim_{n\to\infty}\pa{f_n(x) - f_m(x)} < 
  \lim_{n\to\infty} \epsilon\\
  -\epsilon &< 
  f(x) - f_m(x) < 
  \epsilon
\end{align*}
so we can conclude 
\begin{equation*}
  \abs{
    f_m(x) - f(x)
  } < \epsilon
\end{equation*}
for this $N$ we chose in the beginning, and we have shown uniform convergence.
}

\bx{
We have that 
\begin{align*}
  \abs{
    f(x) - f(y)
  }  
  &= \abs{
    f(x) - f_n(x) + f_n(x) - f_n(y) - (f(y) - f_n(y))
  }\\
  &\leq 
  \abs{
    f(x) - f_n(x)
  } +
  \abs{
    f_n(x) - f_n(y) 
  } + 
  \abs{
    f(y) - f_n(y)
  }\\
  &<
  \frac{\epsilon}{3} +
  \frac{\epsilon}{3} +
  \frac{\epsilon}{3}\\
  &= \epsilon
\end{align*}
We know from the uniform convergence $f_n \rightarrow f$ that 
we can find $N_x, N_y$ such that for $n_x \geq N_x, n_y \geq N_y$, that
\begin{equation*}
  \abs{f_{n_x}(x) - f(x)} < \epsilon/3,
  \abs{f_{n_y}(y) - f(y)} < \epsilon/3
\end{equation*}
Choose $N = \max\pa{N_x, N_y}$.
Then, from the uniform convergence of every $f_n$ for fixed $n \geq N$,
with 
$\exists \delta, \abs{x - y} < \delta$, that
\begin{equation*}
  \abs{
    f_n(x) - f_n(y)
  } < \epsilon/3
\end{equation*}

Choose this $\delta$, after finding $N$, and our proof is complete.
}

\bx{
\ea{
\item We saw that 
\begin{equation*}
  f_n(x) = \frac{x^2 + nx}{n}
\end{equation*}
converges pointwise on $\mathbb{R}$, a compact set, to $x$, but 
is not uniformly continuous, since larger $x$ values require 
larger $N$'s to bound.

\item Yes, we just need to account for the bounded value of $g$.
\item We could consider $f_n(x) = nx^2$ over the interval $[-1, 1]$.
Each $\abs{f_n(x)} \leq n$, but as $n \to \infty$, this is no longer 
bounded. However, this example doesn't work since $nx^2$ does not 
converge to a limit function.

To show that $f$ is bounded if it converges uniformly,
since we can pick some fixed $\epsilon > 0$, and find $N$ such that 
\begin{equation*}
  \abs{
    f_N(x) - f(x)
  } \leq \epsilon
\end{equation*}
Now, we know that $f_N(x) \leq M$ for all $x$ if $N$ is fixed, so therefore
we can conclude that 
\begin{equation*}
  \abs{
    f(x)
  } \leq \epsilon + M,
\end{equation*}
for all $x$, and therefore $f$ is bounded.

\item Yes, we can choose the larger of $N_a, N_b$ to be our $N$, and 
it will satisfy the uniform convergence criteria.

\item We know that for $y > x$
\begin{equation*}
  \forall n, f_n(y) > f_n(x)
\end{equation*} 
now, we can use the Order Limit Theorem, to get 
\begin{align*}
  \lim_{n \to \infty} f_n(y) &>
  \lim_{n \to \infty} f_n(x)\\
  f(y) &>
  f(x) \tag{Since $f_n \to f$}\\
\end{align*}
\label{chap6:increasing_uniform}
\item Our proof in part 
\ref{chap6:increasing_uniform}
did not require uniform convergence, just convergence,
so we can use the same proof.
}
}

\bx{
Since $K$ is compact, $g$ is bounded over $K$.
Now, 
\begin{align*}
  \abs{
    \frac{f_n(x)}{g(x)}
    -\frac{f(x)}{g(x)}
  } = \abs{
    1/g(x)
  }\abs{
    f_n(x) - f(x)
  } < \epsilon
\end{align*}
If we have $g(x)$ bounded, we can say $1/g(x) \leq M'$
then we can choose $N$ such that for $n \geq N$,
\begin{equation*}
  \abs{
    f_n(x) - f(x)
  } < \epsilon/M'
\end{equation*}
}

\bx{
For any $\epsilon > 0$, we know $\exists \delta$ 
such that for $\abs{x - y} < \delta$, $\abs{f(x) - f(y)} < \epsilon$.
We can choose $1/n < \delta \Rightarrow n > 1/\delta$, to show that
\begin{equation*}
  \abs{
    f_n(x) - f(x)
  } = 
  \abs{
    f(x + 1/n) - f(x)
  } < \epsilon,
\end{equation*}
since $\abs{\pa{x + \frac{1}{n}} - x} = \abs{\frac{1}{n}} < \delta$.

Now, with just normal continuity,
we can choose a function like $f(x) = x^2$, which will give us 
\begin{equation*}
  \abs{
    f(x + 1/n) - f(x)
  } = \abs{
    2x/n + 1/n^2
  }
\end{equation*}
which grows bigger as $x$ gets bigger.
}

\bx{
\ea{
\item This is a straightforward application of the triangle inequality,
\begin{equation*}
  \abs{
    \pa{
      f_n(x) + g_n(x)
    } -
    \pa{
      f(x) + g(x)
    }
  } \leq 
  \abs{
    f_n(x) - f(x)
  } +
  \abs{
    g_n(x) - g(x)
  } < 
  \frac{\epsilon}{2} +
  \frac{\epsilon}{2} = \epsilon
\end{equation*}

\item Try $f_n(x), g_n(x) = \frac{1}{x} + \frac{1}{n}$,
then we run into issues with bounding because their limit function 
is $f(x)g(x) = \frac{1}{x^2}$,
but we have troubles with 
\begin{equation*}
  \abs{
    \pa{
      \frac{1}{x^2} + \frac{2}{xn} + \frac{1}{n^2}
    } - 
    \frac{1}{x^2}
  } = \abs{
    \frac{2}{xn} + \frac{1}{n^2}
  },
\end{equation*}
which grows as $x \to 0$.

\item This is the trick where you add and subtract $f_n(x)g(x)$
inside the absolute value, take the triangle inequality, and bound.
The only tricky part is that we need to bound $\abs{g(x)}\abs{f_n(x) - f(x)}$,
but we can do this because $g_n$ is bounded for all $n$, and 
we know that $g_n$ is uniformly convergent.
}
}

\bx{
\ea{
\item Fix some $x \in K$.
Since $f_n$ is increasing in $n$, $g_n(x) = f(x) - f_n(x)$ is decreasing 
in $n$.
We also have that $g_n(x)$ is continuous, since it is the subtraction of 
two continuous functions, $f, f_n$.
In addition, we have that $g_n(x) \to 0$, since $f_n \to f$.

\item 
This means for any $x \in K_j$, since we have $g_j(x) \geq \epsilon$,
it must be the case that for $i < j, g_i(x) \geq \epsilon$ as well, 
since $g_i(x) \geq g_j(x)$.
Therefore, $x \in K_i$.

This shows that 
\begin{equation}
  K_i \supseteq K_j \Rightarrow K_1 \supseteq K_2 \supseteq K_3 \supseteqq \cdots
  \label{chap6:eq:nested_k}
\end{equation}

We also want to have that every $K_n$ is compact,
so we can use the Nested Interval Property.
For any sequence $x_m \to x$,
since $g$ is continuous, we must have 
\begin{equation*}
  \lim_{x_m \to x} g_n(x_m) = g_n(x)
\end{equation*}
By the properties of $K_n$, since $g_n(x_m) \geq \epsilon$,
we know by the Order Limit Theorem that $g_n(x) \geq \epsilon$
as well. This means that $x \in K_n$, by definition of $K_n$.
Therefore, since $K_n$ has all of its limit points, it is a closed set.
Since $K_n \subseteq K_{n-1}$, which is compact (induction...), and therefore bounded,
we conclude that $K_n$ is closed and bounded, i.e. compact.

Now, from Equation 
\ref{chap6:eq:nested_k},
if we \AFSOC that every $K_n \neq \emptyset$,
we know that from the NIP that $\exists x \in K_n$ for all $n$, 
which means $g_n(x) \to \epsilon_ 0 \geq \epsilon > 0$.
But this is a contradiction, since $g_n(x) \to 0$.

So therefore we must have that $\exists K_n = \emptyset$,
which means 
\begin{equation*}
  \exists n, g_n(x) < \epsilon \Rightarrow \abs{f(x) - f_n(x)} < \epsilon
\end{equation*}
for all $x$, which means that $f_n \to f$ uniformly.
}
}

\clearpage
\bx{
\ea{
\item Sketches
\begin{figure}[H]
  \centering
  \def\coordinateWidth{4}
  \def\tickWidth{2}
  \subfloat[$f_0(x)$]{
    \begin{tikzpicture}
      \draw
        (0, 0) -- (0, \coordinateWidth)
        (0, 0) -- (\coordinateWidth, 0)
        (0, 0) node[below left] {$0$}
      ;

      \draw[->]
        (0, 0) -- (\coordinateWidth, \coordinateWidth)
      ;

      \draw
        (\coordinateWidth, \tickWidth pt) --
        (\coordinateWidth, - \tickWidth pt)
        node[below] {$1$}
        (\tickWidth pt, \coordinateWidth) --
        (-\tickWidth pt, \coordinateWidth)
        node[left] {$1$}
      ;

      %% TODO: probably not good style but an easy fix
      \draw[white]
        (\coordinateWidth/2  , \tickWidth pt) --  
        (\coordinateWidth/2  , -\tickWidth pt)
        node[below] {$1/2$}
      ;
    \end{tikzpicture}
  }
  \qquad
  \subfloat[$f_1(x)$]{
    \begin{tikzpicture}
      \draw
        (0, 0) -- (0, \coordinateWidth)
        (0, 0) -- (\coordinateWidth, 0)
        (0, 0) node[below left] {$0$}
      ;

      \draw[->]
        (0, 0) -- 
        (\coordinateWidth/3, \coordinateWidth/2) --
        (\coordinateWidth*2/3, \coordinateWidth/2) --
        (\coordinateWidth, \coordinateWidth)
      ;

      \draw
        (\coordinateWidth/3  , \tickWidth pt) --  
        (\coordinateWidth/3  , -\tickWidth pt)
        node[below] {$1/3$}
        (\coordinateWidth*2/3, \tickWidth pt) --
        (\coordinateWidth*2/3, - \tickWidth pt)
        node[below] {$2/3$}
        (\coordinateWidth, \tickWidth pt) --
        (\coordinateWidth, - \tickWidth pt)
        node[below] {$1$}
        (\tickWidth pt, \coordinateWidth/2) --
        (-\tickWidth pt, \coordinateWidth/2)
        node[left] {$1/2$}
        (\tickWidth pt, \coordinateWidth) --
        (-\tickWidth pt, \coordinateWidth)
        node[left] {$1$}
      ;
    \end{tikzpicture}
  }
  \caption{Sketches for $f_n(x), n=0, 1$}
  \label{chap6:fig:plot_cantor_fx}
\end{figure}

\item Sketching $f_2(x)$,
\begin{figure}[H]
  \centering
  \def\coordinateWidth{6}
  \def\tickWidth{2}
  \begin{tikzpicture}
    \draw
      (0, 0) -- (0, \coordinateWidth)
      (0, 0) -- (\coordinateWidth, 0)
      (0, 0) node[below left] {$0$}
    ;

    \draw[->]
      (0, 0) -- 
      (\coordinateWidth/9, \coordinateWidth/4) --
      (\coordinateWidth*2/9, \coordinateWidth/4) --
      (\coordinateWidth/3, \coordinateWidth/2) --
      (\coordinateWidth/3, \coordinateWidth/2) --
      (\coordinateWidth*2/3, \coordinateWidth/2) --
      (\coordinateWidth*7/9, \coordinateWidth*3/4) --
      (\coordinateWidth*8/9, \coordinateWidth*3/4) --
      (\coordinateWidth, \coordinateWidth)
    ;

    \draw
      (\coordinateWidth/3  , \tickWidth pt) --  
      (\coordinateWidth/3  , -\tickWidth pt)
      node[below] {$1/3$}
      (\coordinateWidth*2/3, \tickWidth pt) --
      (\coordinateWidth*2/3, - \tickWidth pt)
      node[below] {$2/3$}
      (\coordinateWidth, \tickWidth pt) --
      (\coordinateWidth, - \tickWidth pt)
      node[below] {$1$}
      (\tickWidth pt, \coordinateWidth/2) --
      (-\tickWidth pt, \coordinateWidth/2)
      node[left] {$1/2$}
      (\tickWidth pt, \coordinateWidth) --
      (-\tickWidth pt, \coordinateWidth)
      node[left] {$1$}
    ;
  \end{tikzpicture}
  \caption{Sketch for $f_2(x)$}
  \label{chap6:fig:plot_cantor_fx_2}
\end{figure}

\item For any $m < n$, the difference between $f_m(x)$ and $f_n(x)$
will be at the first time $f_m$ is a upward sloping line, while $f_n$
potentially has more ``kinks'' in it.
\begin{figure}[H]
  \centering
  \def\coordinateWidth{4}
  \def\tickWidth{2}
  \begin{tikzpicture}
    \draw
      (0, 0) -- (0, \coordinateWidth)
      (0, 0) -- (\coordinateWidth, 0)
      (0, 0) node[left] {$y_0$}
    ;

    \draw
      (0, 0) -- 
      (\coordinateWidth/3, \coordinateWidth/2) --
      (\coordinateWidth*2/3, \coordinateWidth/2) --
      (\coordinateWidth, \coordinateWidth)
    ;
    \draw
      (0, 0) -- 
      (\coordinateWidth, \coordinateWidth)
    ;

    \draw
      (\tickWidth pt, \coordinateWidth) --
      (-\tickWidth pt, \coordinateWidth)
      node[left] {$y_0 + \frac{1}{2^m}$}
    ;
  \end{tikzpicture}
  \caption{Difference between $f_m, f_n$ at kinks}
  \label{chap6:fig:kink_diff}
\end{figure}
But what we notice is despite the kinks, since they happen between these 
upward lines of $f_m$, which are at most height $\frac{1}{2^m}$,
we can say that 
\begin{eqnarray}
  \abs{
    f_m(x) - f_n(x)
  } \leq \frac{1}{2^m}.
\end{eqnarray}
This means for any $\epsilon > 0$, we can find $\frac{1}{2^m} < \epsilon$,
and show the Cauchy Criterion for Uniform Convergence for function sequences,
which implies uniform convergence for $f_n$.

\item Every $f_n$ is continuous and increasing, and we showed that
$f_n \to f$ uniformly, so therefore $f$ is also continuous and increasing.

We have $f'(x) = 0$ $\forall x \in [0, 1] \setminus C$, since 
these are the portions where the function is flat.
}
}

\bx{
\ea{
\item $f_n(x_1)$ contains a convergent subsequence because $f_n(x)$ is bounded,
and this is a direct application of the Bolzano-Weierstrass Theorem.

\item Same reason as above, we stil have a bounded sequence on reals, so 
we can still apply Bolzano-Weierstrass.

\item We can construct $f_n$ by the following 
\begin{equation*}
  f_n(x) = \text{First element of } f_{n, k}
\end{equation*}
since every $f_{n, k} \in f_{m, k}$
for $m < n$, we conclude that for any $x_m \in A$,
$f_n(x_m) \in f_{m, k}$ after $n > m$, and thus 
this $f_n(x_m)$ for $n > m$ is a convergent subsequence,
and therefore $f_n$ converges at every point of $A$.
}
}

\bx{
\ea{
\item The difference is that we need a single $\delta$ that works for $\abs{x - y} < \delta$,
whereas if we say for each $f_n$ is uniformly continuous, we can 
choose a different $\delta$ for each $f_n$.
This equicontinuous definition is stronger.

\item We saw that as $n \to \infty$, $x_n$ has a jump at $1$, where 
for $x < 1$, $x \to 0$, and $x = 1$ if $x = 1$.
This means for any $\epsilon, \delta$, we can always choose a large enough 
$n$ so that the $\delta$-neighborhood is not small enough.

Each $g_n$ is uniformly continuous, since we can say 
\begin{align*}
  \abs{
    g_n(x) - g_n(y)
  } &= \abs{
    x^n - y^n
  }\\ 
  &= \abs{
    (x-y)\pa{
      \sum_{i=1}^n x^{n-i}y^{i}
    }
  }\\
  &\leq \abs{x-y}n\tag{Since $x, y\in [0, 1], x^ky^k \leq 1$}\\
  \Rightarrow \delta &< \frac{\epsilon}{n} \tag{Will win the $\epsilon$-challenge}
\end{align*}
}
}

\bx{
\ea{
\item $\mathbb{Q}$ is countable, so we can find this 
$g_k = f_{n_k}$ that converges at every $q\in \mathbb{Q}$ on $[0, 1]$.

\item We know for every $r_i$, $\exists N_i, s, t \geq N_i$ such that 
\begin{equation*}
  \abs{
    g_s(r_i) - g_t(r_i)
  } < \frac{\epsilon}{3},
\end{equation*}
since we know $g_k$ converges so all $r \in \mathbb{Q}$ on $[0, 1]$,
so we can take $N = \max_{i=1}^m\pa{N_i}$.

We need $\pbrac{r_i}_{i=1}^m$ to be finite since we wanted to take
the max $N$ from $\pbrac{N_i}_{i=1}^m$, which may not exist if 
the set is not finite.

\item We can show for any $x \in [0, 1]$, choose $r \in \mathbb{Q}$
such that $\abs{x - r} < \delta$, 
then
\begin{align*}
  \abs{
    g_s(x) - g_t(x)
  } 
  &= \abs{
    \pa{
      g_s(x) - g_s(r)
    } + 
    \pa{
      g_s(r) - g_t(r)
    } - 
    \pa{
      g_t(x) - g_t(r)
    }
  }\\
  &\leq \abs{
    g_s(x) - g_s(r)
  } + \abs{
    g_s(r) - g_t(r)
  } + \abs{
    g_t(x) - g_t(r)
  }\\
  &< \frac{\epsilon}{3} +
  \frac{\epsilon}{3} + 
  \frac{\epsilon}{3}
  = \epsilon
\end{align*}
}
}