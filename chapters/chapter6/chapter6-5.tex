%% 6.5 %%

\section{Power Series}
\setcounter{exercise}{0}

\bx{
\ea{
\item $g$ is defined on $(-1, 1)$, since we can 
bound 
\begin{equation*}
  \abs{
    \frac{x^n}{n}
  } \leq x^n \leq y^n
\end{equation*}
for $y$ that is in between $\pm 1$ and $x$.

$g$ converges uniformly on this set, and every $g_n$ is continuous,
so $g$ is continuous.

$g$ is still defined on $(-1, 1]$, since at $x=1$, 
we have the alternating harmonic series, which converges.

For $x = -1$, we have 
\begin{equation*}
  g(-1) = -1 - 1/2 - 1/3 - 1/4 - \cdots
\end{equation*}
which diverges.

For $\abs{x} > 1$, $g$ will diverge, since $x^n/n$ is increasing.

$g(x)$ cannot converge for other $x$, since we know that if $g$ did for $\abs{x} > 1$,
that would imply $x =-1$ has to converge, which we know is not the case.

\item $g'$ is defined for $x \in (-1, 1)$. 
\begin{equation*}
  g'(x) = 1 - x + x^2 - x^3 + x^4 - \cdots = \frac{1}{1+x}
\end{equation*}
we can use the geometric series formula since $\abs{x} < 1$.
}
}

\bx{
\ea{
\item $a_n = 1/n^2$.
\item $a_0 = 0$, for $n\geq1, a_n = \frac{1}{n}$. 
This will create the alternating harmonic series for $x = -1$, but
the harmonic series for $x = 1$.
\item You can define 
\begin{equation*}
  a_n = \begin{cases}
    0, &n=0, n \text{ odd}\\
    \frac{-1^{n/2}}{n/2}, &n\geq 2, n \text{ even}
  \end{cases}
\end{equation*}
The idea is that $x=-1$ will alternate between $-1, 1$ for $x^n$,
but will always be $1$ if $x^{2k}$, so we can easily make a conditionally 
convergent sequence for the even terms, e.g.
\begin{align*}
  x=-1 : \,&a_0(1)+ a_1(-1)+ a_2(1)+ a_3(-1)+ a_4(1)+ \dots\\
  x=1  : \,&a_0(1)+ a_1(-1)+ a_2(1)+ a_3(-1)+ a_4(1)+ \dots
\end{align*}

\item No, because if the power series converges absolutely at $x=1$,
we have 
\begin{equation*}
  \abs{a_n(-1)^n} = \abs{a_n(1)^n}
\end{equation*}
which means the $x=-1$ series will also converge absolutely.
}
\label{chap6:power_series_examples}
}

\bx{
\AFSOC we have conditional convergence at 3 points, $x_1, x_2, x_3$.
Since we have 3 different points, if we look at their absolute values,
it must be the case that at least one of the points has a different absolute value 
than the others. Let this point be $x_3$ WLOG.

We have 2 cases,
\begin{itemize}
  \item $\abs{x_3} < \abs{x_1}$. Since the series converges at $\abs{x_1}$,
  it must converge on $S = \pbra{-\abs{x_1}, \abs{x_1}}$, which means since 
  $x_3 \in S$, that $x_3$ must converge absolutely.

  \item $\abs{x_3} > \abs{x_1}$. Since the series converges at $\abs{x_3}$,
  it must converge on $S = \pbra{-\abs{x_3}, \abs{x_3}}$, which means since 
  $x_1 \in S$, that $x_1$ must converge absolutely.
\end{itemize}

In all cases, we show that some point out of the 3 converges absolutely,
so therefore at most two points can converge conditionally.

We have shown in Exercise
\label{chap6:power_series_examples}
an example where we have conditional convergence at two points $x_1, x_2$.
}

\bx{
\ea{
\item For any $x \in (-R, R)$, choose $c \in (\abs{x}, R)$.
Then since the power series converges at $c$, it must converge 
absolutely at $x$, since $\abs{x} < c$, from Theorem 6.5.1.
Theorem 6.5.2 then tells us that on $[-\abs{x}, \abs{x}]$,
the power series will converge uniformly.
From Theorem 6.4.2, if $\sum_{n=1}^\infty f_n$ converges uniformly
to $f$ on $A$, then $f$ is continuous on $A = \pbra{-\abs{x}, \abs{x}}$,
and we know $x \in A$.

\item By Theorem 6.5.1, we know that the power series will 
converge at any $\abs{x} < R$, which is the set $(-R, R)$,
so our convergence is 
\begin{equation*}
  \pa{-R, R} \cup \pbrac{R} = \left(-R, R\right]
\end{equation*}
}
}

\bx{
If we have absolute convergence at a point $x_0$, 
then we know $\sum \abs{a_nx_0^n}$ converges.

Then for any $x \in \pbra{-\abs{x_0}, \abs{x_0}}$, we have 
\begin{equation*}
  M_n = \abs{a_nx_0^n} \geq \abs{a_nx^n}
\end{equation*}
so therefore by the Weierstrass M-Test, we can conclude that 
this $\sum a_nx^n$ converges.
}

\bx{
Take some compact set $K \subseteq A$.
We know that $K$ must be bounded, since it is compact, so say 
$K \subseteq \pbra{-M, M} \subseteq A$.

We know the power series converges at $M$, since $M \in A$.
Now, by Abel's Theorem, we have that since the power series 
converges at $M$, we conclude the series converges uniformly
on $[0, M]$. Similarly, it converges at $-M$, so 
the power series converges uniformly on $[-M, 0]$ as well.

Since the power series converges uniformly on 
\begin{equation*}
  \pbra{-M, 0} \cup \pbra{0, M} \supseteq K,
\end{equation*}
we conclude the series converges uniformly on $K$ as well.
\label{chap6:uniform_compact_conv}
}

\bx{
\ea{
\item We have
\begin{equation*}
  \lim_n \abs{
    \frac{
      (n+1)s^{n}
    }{
      ns^{n-1}
    }
  } = 
  \lim_n \abs{
    \frac{n+1}{n}s
  } = 
  \abs{s} < 1
\end{equation*}
\label{chap6:ratio_conv}

\item If we have $\sum_{n=0}^\infty a_nx^n$ converging on 
$x \in (-R, R)$, we can use Theorem 6.5.5 from Exercise 
\ref{chap6:uniform_compact_conv}
to show that this series also converges uniformly on compact sets
contained in $(-R, R)$.

Now for the differentiated series, 
choose a $t \in (\abs{x}, R)$,
we can write 
\begin{equation*}
  \sum \abs{na_nx^{n-1}} = 
  \sum \frac{1}{t} \pa{
    n \abs{\frac{x}{t}}^{n-1}
  }\abs{a_nt^n}
\end{equation*}
Now, since we know $\sum n\abs{x/t}^{n-1} = L$ because it converges, 
from part
\ref{chap6:ratio_conv},
we can now write 
\begin{equation*}
  \sum \abs{na_nx^{n-1}} \leq
  \sum \frac{1}{t} \pa{
    L
  }\abs{a_nt^n}
  = \frac{L}{t} \sum \abs{a_nt^n} = \frac{LL'}{t}
\end{equation*}
we know $\sum \abs{a_nt^n}$ converges since we know
the power series converges at $t' \in (t, R)$.

Therefore, the differentiated series also converges.
Using Theorem 6.5.5 again, we can conclude the differentiated
series also converges uniformly on compact sets in $(-R, R)$.
}
}

\bx{
\ea{
\item We can use the ratio test,
\begin{equation*}
  \lim_n \abs{
    \frac{a_{n+1}x^{n+1}}{a_nx^n}
  } = 
  L \abs{x} < 1.
\end{equation*}
which shows that the series converges.
\label{chap6:ratio_test_an}

\item We can use the ratio test as in part
\ref{chap6:ratio_test_an}.

\item If $L' = \sup\pbrac{\abs{a_{k+1}/a_k}}$, 
then $L \leq L'$, so for $x \in (-1/L', 1/L')$,
we apply the ratio test again,
\begin{equation*}
  \lim_n \abs{
    \frac{a_{n+1}x^{n+1}}{a_nx^n}
  } = 
  L \abs{x} \leq L' \abs{x} < 1.
\end{equation*}
If $L' = 0$, this is not possible since that would imply $a_k = 0$
for all $k$, but we know $a_k \neq 0$.

\item When $x = 0$, this series $=0$, so it converges trivially.
When $x \neq 0$, this proof is more involved.
If we have $\abs{x} \geq 1$, this proof is not too hard.
We can show $\abs{\sum a_nx^n}$ does not converge, 
because we can choose some $M > 1$, and then 
we can find $n$ such that $\abs{a_{n+1}/a_n} \geq M$.
This means 
\begin{equation*}
  \abs{
    a_{n+1}x^{n+1} - a_nx^n
  } \geq \abs{
    a_{n+1}x^{n+1}
  } - \abs{
    a_nx^n
  } = 
  (M-1)\abs{a_{n-1}}
  \pa{
    \abs{x}-1
  }
  \abs{
    x^{n-1}
  } > \epsilon_1\abs{a_{n-1}}\delta_1 \cdot 1 > 0
\end{equation*}
which means after this $n$, the sum keeps on oscillating with 
an amplitude that is always greater than 0, which means it 
cannot possibly converge.

For $x \in (-1, 1)$, what we need to show is somehow that $\abs{a_{n+1}/a_n}$
``grows faster'' than $x^n$ decreases.
Otherwise, it is possible that $x^n$ can diminish the unboundedness of the
coefficients.
The author probably did not intend for this proof, 
so I will not prove it. It might be not too hard to show, 
but I don't feel like doing it...

The solution says that the condition should be $\abs{a_{k+1}/a_k} \geq 1$
after some $k \geq N$, but I don't think this is necessarily true, because if you take 
some $\abs{x} < 1$, and $a_n = 1$, then the series will still converge.
}
}

\bx{
We can show that 
\begin{equation*}
  \sum_{n=0}^\infty a_nx^n -
  \sum_{n=0}^\infty b_nx^n
  = 
  a_0 - b_0
  =
  0
\end{equation*}
so therefore $a_0 = b_0$.
We can inductively continue by considering
for $x \neq 0$,
\begin{equation*}
  \frac{1}{x}
  \pa{
    \sum_{n=1}^\infty a_nx^n -
    \sum_{n=1}^\infty b_nx^n
  }
  = 
  a_1 - b_1
  =
  0 \Rightarrow a_1 = b_1
\end{equation*}
So therefore $\forall a_n = b_n$, and we conclude the power series 
is unique.
}

\bx{
We are given that 
\begin{equation*}
  \sum a_n, \sum b_n, \sum d_n
\end{equation*}
all converge. 
Notice that 
\begin{equation*}
  f(1) = \sum a_n 1^n = \sum a_n,
\end{equation*}
and that similar observations can be made about $g, h$.

Then, the assumptions mean that $f(1), g(1), h(1)$
all converge, and by Abel's Theorem, this means that 
$f, g, h$ converge uniformly on $[0, 1]$, and thus are also 
continuous on this set.

Now, we also know that since $f, g, h$ converge at 1, that 
they converge absolutely for any $x \in [0, 1)$, 
which means in this range, we have that 
\begin{equation*}
  h(x) = f(x)g(x)
\end{equation*}
Therefore, since $f, g, h$ are continuous on $[0, 1)$,
and they also converge at $1$, we can conclude that at 
$x=1$, we also have 
\begin{equation*}
  h(1) = f(1)g(1) \Rightarrow \sum d_n = \pa{\sum a_n}\pa{\sum b_n} = AB
\end{equation*}
}

\bx{
\ea{
\item Any series $\sum a_n = L$ means that 
\begin{equation*}
  f(x=1) = \sum a_n1^n = L
\end{equation*}
so $f(x)$ converges uniformly on $[0, 1]$ by Abel's Theorem.
Since all $a_nx^n$ are continuous on $[0, 1]$, $f(x)$ 
is also continuous over this set, so we can conclude
$\lim_{x \to 1^-} f(x) = f(1) = L$.

\item This is just the geometric series with $b_0 = 1$, $r = -1 \cdot x$,
so we have 
\begin{equation*}
  \sum_{n=0}^\infty a_nx^n = \frac{1}{1-(-x)} = \frac{1}{1+x}
\end{equation*}
If we take this $\lim_{x\to 1^-}$, we get a limit of $L = 1/2$.
}
}

\bx{
The key observation here is that $G(0) = d_1$, 
and $G$ is strictly increasing.

This means that $d_{r+1} = G(d_r) \leq G(d_0)$, 
so by the Order Limit Theorem, we conclude $G(d_r) \to G(d_0) = d_0$.
So $\lim d_r = d_0$.
}