%% 2.6 %%
\section{The Cauchy Criterion}
\setcounter{exercise}{0}

\bx{
\ea{
	\item $a_n = 1 + \pa{-\frac{1}{2}}^n$
	\item $a_n = n$
	\item Impossible, since a Cauchy sequence implies convergence, which means every subsequence will also converge.
	\item You can use Equation  (\ref{eq:convdiv}). Literally anything that diverges but has a convergent subsequence.
}
}

\bx{
If we have that $(x_n) \rightarrow x$, then we can make $\abs{x_n - x}$ arbitrarily small. Consider
\begin{align*}
	\abs{x_n - x_m} 
	&= \abs{x_n - x + x - x_m} \\
	&\leq \abs{x_n - x} + \abs{x_m - x} \tag{Triangle Inequality} \\
	&< \frac{\epsilon}{2} + \frac{\epsilon}{2} = \epsilon
\end{align*}
}

\bx{
\ea{
	\item The \textit{pseudo-Cauchy} definition is different because it only looks at consecutive terms
	\item Consider the harmonic series, where $\abs{s_{n+1} - s_{n}} = \frac{1}{n(n+1)}$.
}
}

\bx{
\begin{align*}
	\abs{c_{n+1} - c_n} &= \abs{\abs{a_{n+1} - b_{n+1}} - \abs{a_n - b_n}} \\
	&\leq \abs{a_{n+1} - a_n + b_{n+1} - b_n} \\
	&\leq \abs{a_{n+1} - a_n} + \abs{b_{n+1} - b_n} \\
	&< \frac{\epsilon}{2} + \frac{\epsilon}{2} = \epsilon.
\end{align*}
}

\bx{
\ea{
\item Let $a_n = x_n + y_n$, then
\begin{align*}
	\abs{a_{n+1} - a_n} &= \abs{x_{n+1} - x_n + y_{n+1} - y_n} \\
	&\leq \abs{x_{n+1} - x_n} + \abs{y_{n+1} - y_n} \\
	&< \epsilon \tag{for proper choice of $N$}
\end{align*}
\item Let $a_n = x_ny_n$, then
\begin{align*}
	\abs{a_{n+1} - a_n} 
	&= \abs{x_{n+1}y_{n+1} - x_ny_n} \\
	&= \abs{x_{n+1}y_{n+1} - x_ny_{n+1} - x_ny_n + x_ny_{n+1}} \\
	&\leq \abs{y_{n+1}\pa{x_{n+1} - x_n}} + \abs{x_n\pa{y_{n+1} - y_n}}
\end{align*}
we have shown we can bound this before by 
\begin{itemize}
	\item $(x_n), (y_n)$ are convergent sequences, so they must be bounded. Call this bound $M$
	\item Now, $\abs{x_{n+1} - x_n}$ can be made arbitrarily small. Given some $\epsilon$, we can make it $< \frac{\epsilon}{2M}$.
	\item We do the same for $\abs{y_{n+1} - y_n}$, and thus the overall bound is $< \epsilon$.
\end{itemize}
}
}

\bx{
I'm not going to write these down super rigorously, but will write down most of the ideas.
\ea{
\item Suppose we have some set of real numbers that is bounded above. We want to show that 
there exists a least upper bound, assuming the Nested Interval Property is true.

Let $B$ be the set of upper bounds. Define $I_1 = B$, and for each subsequent $I_i$, define it as 
\begin{equation*}
	I_{n+1} = \{b \in I_n : b < \ell_{n+1}\},
\end{equation*}
where $\ell_{n+1} \in I_n$ is arbitrarily chosen. The idea is that we are creating intervals 
that have a smaller and smaller maximum value.

Now we have 2 cases,
\begin{enumerate}[label=\alph*]
	\item $\exists n \in \mathbb{N} : I_n = \emptyset$. In this case, there must have been 
	some $\ell_n$ where $\forall b \in I_n, b \neq \ell_n, \ell_n < b$. Since $I_n$ is a subset of
	the smaller elements of $B$, we also have $\forall b \in B, b \neq \ell_n, \ell_n < b$,
	which means this $\ell_n$ is the least upper bound.
	\item None of the $I_n$ are empty. Now, consider $\bigcap_{n=1}^\infty I_n$.
	By NIP, this intersection is nonempty. By our construction of the $I_n$, we know any element $b$
	in this intersection must be less than all the elements in the set preceding it.
	However, we reach a contradiction, because if $b$ is in this infinite intersection, 
	\TODO this proof doesn't work... probably need to change the interval construction.
\end{enumerate}

\item Define $i_n$ to be $\inf \pa{\bigcup_{k=n}^\infty I_k}$, and $s_n$ to be $\sup \pa{\bigcup_{k=n}^\infty I_k}$.
$i_n$ is an increasing sequence, and $s_n$ is a decreasing sequence.
Since for any set, $\inf A \leq \sup A$, we know that $\forall n : i_n \leq s_n$.
This means $\exists x : \lim i_n \leq x \leq \lim s_n$, which exists in every single interval $I_k$,
so we know their infinite intersection is nonempty.

\item We are using the BW Theorem to prove NIP. 
Let $x_i$ be an arbitrary element of $I_i$.
since $I_1$ is bounded, then all $I_k, k\geq 1$ are also bounded, 
so the sequence $(x_n)$ is also bounded.
By BW, we know that $\exists (s_n)$, a subsequence of $(x_n)$ that converges.
Suppose $\lim s_n = L$. We will show that $L$ is in the infinite intersection of all the sets.
For any $I_k$, the interval is $(a_k, b_k)$. Call $\epsilon = \abs{a_k - b_k}$, the size of the interval.
Since $(s_n)$ converges to $L$, we know $\exists N : n \geq N$,
\begin{equation*}
	\abs{s_n - L} < \epsilon/2.
\end{equation*} 
since $s_n \in I_n$ by definition, we see $L$ also falls in this interval. Thus, we know 
$L \in I_n$ for $n \geq N$. Since $I_n$ is also contained within all sets before as well, $L$ also 
is contained in those sets. Therefore, we have shown that $L$ is in every set, so the infinite intersection must 
contain at least $L$, so therefore it is nonempty.

\item We are given a bounded sequence, and want to show that there exists a 
convergent subsequence. 

We will construct a subsequence, and show that it is convergent.

Since we have a bounded sequence $(x_n)$, we know $\abs{x_n} \leq M$.

Let $I_1 = [-M, M]$. Now construct subsequent $I_n$ from $I_{n-1} = [a, b]$ as 
\begin{equation*}
	I_n = [a, (a+b)/2] \text{ or } [(a+b)/2, b],
\end{equation*}
depending on which half has an infinite number of elements.

Now, our subsequence is defined as $a_n : a_n \in I_n$. 
Given $\epsilon > 0$, since we can make the bound of $I_n$ 
as arbitrarily small as possible, since we are halving the interval size every time, 
we know $\exists N : n \geq N \rightarrow \abs{I_n} < \epsilon$. 
Now, since all future elements $a_k$ will be chosen from this interval and its subsets,
and the interval size is $< \epsilon$, we can conclude for $m, n \geq N$,
\begin{equation*}
	\abs{a_n - a_m} < \epsilon.
\end{equation*}
This means $(a_n)$ is a Cauchy Sequence, and by the Cauchy Criterion, we can conclude that 
$(a_n)$ is convergent. 
}
}