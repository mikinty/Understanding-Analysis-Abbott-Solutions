%% 4.4 %%

\section{Continuous Functions on Compact Sets}
\setcounter{exercise}{0}

\bx{
\ea{
\item For any $\epsilon > 0, c \in \mathbb{R}$, 
choose $\delta < \sqrt[3]{\epsilon}$.

\item Take $x = n$, $y_n = n + 1/n$. Choose $\epsilon = 1$,
then $x_n, y_n \rightarrow n, so \abs{x_n - y_n}\rightarrow 0$, but 
\begin{equation*}
  \abs{f(x) - f(y)} = \abs{n - \pa(n^3 + 3n + 3/n + 1/n^3)} 
  = \abs{3n + 3/n + 1/n^3}
  > 3 > \epsilon = 1
\end{equation*}

\item For any bounded subset, let $M$ be the largest absolute value of any
element in the subset. Then choose $\delta < \frac{\epsilon}{3M^2}$.
Then we have 
\begin{equation*}
  \abs{x^3 - y^3} 
  = 
  \abs{\pa{
    x - y
  }\pa{
    x^2 + xy + y^2
  }}
  < \frac{\epsilon}{3M^2} \cdot \pa{
    M^2 + M\cdot M + M^2
  }
  = \epsilon
\end{equation*}
}
}

\bx{
Given any $\epsilon$, choose $\delta < \epsilon/2$.
Then for $\abs{x - y} < \delta$,
\begin{align*}
  \abs{f(x) - f(y)} 
  &= \abs{\frac{1}{x^2} - \frac{1}{y^2}}\\
  &= \abs{
    \frac{
      y^2-x^2
    }{
      x^2y^2
    }
  }\\
&= \abs{
  \frac{
    (y-x)(y+x)
  }{
    x^2y^2
  }
}\\
&\leq \abs{
  x-y
}\abs{
  \frac{
    x+y
  }{x^2y^2}
}\tag{Since $\frac{1}{x^2y^2} \leq 1$}\\
&= \abs{
  x-y
}\abs{
  \frac{
    1
  }{x^2y}
  + \frac{1}{y^2x}
}\\
&< \frac{\epsilon}{2}\pa{1 + 1}\tag{$\frac{1}{x^2y} \leq 1$, since $x, y \geq 1$}\\
&= \epsilon
\end{align*}

Now, for $(0, 1]$, choose $x_n = \frac{1}{n}, y_n = \frac{1}{n+\frac{1}{n}}$.
Also choose $\epsilon_0 = 1/2$.
Then $\abs{x_n = y_n} = \abs{\frac{1}{n^3+n}} \rightarrow 0$, but 
\begin{equation*}
  \abs{
    f(x_n) - f(y_n)
  } = \abs{
    n^2 - \pa{n+\frac{1}{n}}^2
  } = 
  \abs{
    -2 - \frac{1}{n^2}
  } \geq 1 > \epsilon_0
\end{equation*}
}

\bx{
Since $K$ is compact, by the preservation of compact sets, since $f$
is continuous on $K$, $f(K)$ is also compact.
Now, we already showed in Exercise \ref{chap3:compact_sup_inf}
that any compact set contains its maximum and minimum values, so therefore
we can conclude that $f$ will attain its maximum and minimum values 
of this range.
}

\bx{
We know $[a, b]$ is closed and bounded, so therefore it is compact. 
Thus, $f([a, b])$ is also compact, since $f$ is continuous on this interval.
Let $M$ be the lower bound of $f([a, b])$, which we know exists since it is
compact, then we can conclude $1/f$ is bounded by $1/M$ on $[a, b]$.
}

\bx{
Choose $\epsilon = \epsilon_0$. Now, we can show 
\begin{equation*}
  \exists m,n \in \mathbb{N}, \abs{x_m - y_m} < \delta_n = \frac{1}{n}
\end{equation*}
such that $\abs{f(x_m)-f(y_m)} \geq \epsilon_0$.
Since we can make $\delta_n$ arbitrarily small, any $\delta$ claim can be shown 
to be faulty, so therefore this function cannot be uniformly continuous.
}

\bx{
\ea{
\item We can take advantage of the fact that $f$ is not necessarily continuous at $0, 1$.
An example is $f(x) = \frac{1}{x}$, where if we take $x_n = \frac{1}{n}$, then $x_n \rightarrow 0$, 
but $f(x) = \frac{1}{1/n} = n$ does not converge.

\item Impossible. Since $[0, 1]$ is closed, any $x_n \rightarrow x$ will have $x \in [0, 1]$.
Since $f$ is continuous over $[0, 1]$, it must also be continuous at $x$, so therefore $\lim f(x_n) = f(x)$,
and therefore $f(x_n)$ is a Cauchy sequence.
\label{chap4:cont_cauchy}

\item Impossible. Same reasoning as 
part \ref{chap4:cont_cauchy}, 
since $[0, \infty)$ is closed, any $x_n \rightarrow x$ will have its limit also in the set.

\item An upside down parabola with a peak in $(0, 1)$ will work. 
A function like $\frac{1}{x(x-1)}$ is a bit more fun.
}
}

\bx{
Suppose we have some $\epsilon > 0$.

Now if we have two arbitrary $x, y \in (a, c)$, if $x, y$ are both in the
left set or the right set, then we can just use the fact that $g$ is
uniformly continuous on those sets to show it. The only other case we have is
that they are in different sets,
WLOG $x \in (a, b], y \in [b, c)$,
\begin{align*}
  \abs{
    f(x) - f(y)
  } = \abs{
    f(x) - f(b) - (f(b) - f(y))
  } \leq \abs{f(x) - f(b)} + \abs{f(y) - f(b)}
\end{align*}
Now, since $b$ is in both sets,
We know for $x_a \in (a, b], y \in [b, c)$,
we can find $\delta_a, \delta_b$ respectively such that 
$\abs{x - b} < \delta_a \Rightarrow \abs{f(x)-f(b)} < \epsilon/2$ and 
$\abs{y - b} < \delta_b \Rightarrow \abs{f(y)-f(b)} < \epsilon/2$.

So we can finish up our proof by choosing $\delta < \min(\delta_a,\delta_b)$, so that 
\begin{equation*}
  \abs{x - y} < \delta \Rightarrow  \abs{f(x) - f(y)}
  \leq \abs{f(x) - f(b)} + \abs{f(y) - f(b)}
  < \frac{\epsilon}{2} + \frac{\epsilon}{2} = \epsilon
\end{equation*}
The last part is motivated by that if $x, y$ are in separate sets, we try to make them as close to $b$
as possible so we can use their uniformly continuous properties around $b$.
\label{chap4:uniform_cont_separate_sets}
}

\bx{
\ea{
\item We can show $f$ is uniformly continuous on $[0, b]$, since it is a compact set.
Then, using the same argument as in Exercise
\ref{chap4:uniform_cont_separate_sets},
we can show that if $f$ is uniformly continuous on 
$[0, b], [b, \infty)$, then $f$ is also uniformly continuous on $[0, \infty)$.
\label{chap4:uniform_cont_separate_sets_closed}

\item We can show $f(x) = \sqrt{x}$ is uniformly continuous on $[1, \infty)$, because 
on this interval, any $x, y \in [1, \infty)$ means $\sqrt{x} + \sqrt{y} \geq 2$. 
Then choose $\delta < 2\epsilon$, then 
\begin{align*}
  \abs{x - y} < \delta \Rightarrow 
  \abs{\sqrt{x}-\sqrt{y}} &= \abs{
    \frac{
      \pa{\sqrt{x}-\sqrt{y}}\pa{\sqrt{x}+\sqrt{y}}
    }{
      \pa{\sqrt{x}+\sqrt{y}}
    }
  }\\
  &= \frac{
    \abs{x-y}
  }{
    \abs{\sqrt{x}+\sqrt{y}}
  }\\
  &< 2 \epsilon \cdot \frac{1}{2} = \epsilon
\end{align*}
Now, $[0, 1]$ is closed and bounded, so it is compact, and $\sqrt{x}$ is
continuous on it, so therefore $\sqrt{x}$ is uniformly continuous on $[0, 1]$.

Using what we just proved in part \ref{chap4:uniform_cont_separate_sets_closed}, 
we can combine these results to show that $\sqrt{x}$ is uniformly continuous
on $[0, 1] \cup [1, \infty) = [0, \infty)$.
\label{chap4:sqrt_function_uniform_cont}
}
\label{chap4:uniform_cont}
}

\bx{
\ea{
\item Choose $\delta < \frac{\epsilon}{M}$, then if we have $\abs{x - y} < \delta = \frac{\epsilon}{M}$,
\begin{equation*}
  \abs{
    \frac{
      f(x) - f(y)
    }{
      x-y
    }
  } \leq M \Rightarrow \abs{f(x)-f(y)} \leq \abs{x-y}M < \frac{\epsilon}{M}M = \epsilon
\end{equation*}

\item No, take $f(x) = \sqrt{x}$. We showed in Exercise \ref{chap4:uniform_cont}, 
part \ref{chap4:sqrt_function_uniform_cont}
that this function is uniformly on $[0, 1]$, but its slope $\frac{1}{2\sqrt{x}}$ tends to $\infty$ 
as $x \rightarrow 0$. The issue stems from that we will have $M = \frac{\epsilon}{\delta}$,
but this quantity may not be bounded, since $\epsilon, \delta$ vary.
}
}

\bx{
Yes, uniform continuousness preserves boundedness over a set.

Choose some $\epsilon > 0$, say $\epsilon = \epsilon_0$.
Now we know since $f$ is uniformly continuous, $\exists \delta \abs{x - y} < \delta \Rightarrow \abs{f(x)-f(y)} < \epsilon_0$.
Now, since $A$ is a bounded set, we can find a finite open cover for it, let this cover
be in the form 
\begin{equation*}
  A = \bigcup_{i=1}^N
    V_{\delta}(x_i)
\end{equation*}
where $x_i \in A$. Now, take $M_1 = \min_{i=1}^N\pa{f(x_i) - \epsilon_0}, 
M_2 = \max_{i=1}^N\pa{f(x_i) + \epsilon_0}$, and $M_1$ will be a lower bound for $f(A)$,
and $M_2$ an upper bound.
}

\bx{
($\Rightarrow$)
Let $O \in \mathbb{R}$ be any open set.
Consider some $x \in g^{-1}(O)$,
We know $g(x) = y \in O$.
Now, since $O$ is an open set, we know $\exists \epsilon$
such that $V_\epsilon(y) \subseteq O$.
We also know that $g$ is continuous, which means for any $\epsilon$,
we know $\exists V_\delta(x)$ such that $x' \in V_\delta(x)$
means $g(x') \in V_\epsilon(y)$. Since $V_\epsilon(y) \subseteq O$,
we have $g(x') \in O$ for any of these $x'$, which means by the definition
of $g^{-1}$, all these $x' \in g^{-1}(O)$, and therefore $V_\delta(x) \subseteq g^{-1}(O)$,
which means there exists an open neighborhood around any element in $g^{-1}(O)$,
and therefore $g^{-1}(O)$ is an open set.

($\Leftarrow$)
If we know $g^{-1}(O)$ is open for any open $O \subseteq \mathbb{R}$,
suppose we get some 
$y \in O, \epsilon > 0$. Then consider the set $V_\epsilon(y) \in O$.
Now, consider $g^{-1}(V_\epsilon(y))$, which we are guaranteed is open.
Since $x = g^{-1}(y)$ is in $g^{-1}(V_\epsilon(y))$, 
because $y \in V_\epsilon(y)$,
we know since $g$ is open, that $\exists \delta > 0$ such that 
$V_\delta(x) \subseteq g^{-1}(V_\epsilon(y))$.
Therefore, for any $V_\epsilon(y)$, we have found a corresponding 
$V_\delta(x)$ such that $x' \in V_\delta(x)$ implies $g(x') \in V_\epsilon(y)$,
which means we conclude $g$ is continuous.
}

\bx{
We want to prove that 
\begin{equation*}
  \textit{
    A function that is continuous on a compact set K 
    is uniformly continuous on K.
  }  
\end{equation*}
% If we have a compact set $K$, we know $\exists$ a finite subcover over $K$.
%% TODO: too hard
\TODO
}

\bx{
\ea{
\item We know for every $\epsilon > 0$, $\exists \delta$ such that
$x' \in V_\delta(x)$ implies $f(x') \in V_\epsilon(f(x'))$.
Now, if we have some Cauchy Sequence $x_n$, we know $\exists N$
such that for $m, n \geq N$,
\begin{equation*}
  \abs{x_m - x_n} < \delta,
\end{equation*}
which means $\abs{f(x_m) - f(x_n)} < \epsilon$, 
since $x_m, x_n \in V_\delta(x)$.
Therefore, we know for this $N$, $m, n \geq N$
means $\abs{f(x_m) - f(x_n)} < \epsilon$, 
so therefore $f(x_n)$ is also a Cauchy Sequence.

\item 
($\Rightarrow$)
We can come up with a sequence $a_n \rightarrow a, a_n \in (a, b)$.
Now, since $(a_n)$ is also a Cauchy Sequence, that means $g(a_n)$ is also a Cauchy sequence,
since $g$ is a uniformly continuous function.
Therefore, define $g(a)$ as $g(a_n) \rightarrow g(a)$. We know this exists since $g(a_n)$ is Cauchy.
Notice that any sequence $(a'_n) \rightarrow a$ is Cauchy, and therefore we must 
have $g(a_n) \rightarrow g(a')$. We want to show that $g(a') = g(a)$ for 
$g$ to be continuous at $a$.

\AFSOC $g(a') \neq g(a)$, then that means we have two sequences $a'_n, a_n \rightarrow a$, 
where these sequences are contained inside $(a, b)$,
but $\abs{g(a') - g(a)} \geq \epsilon_0$, which means $g$ is not uniformly continuous.
This is a contradiction, so therefore we conclude that $g(a') = g(a)$,
and therefore any $(a'_n) \rightarrow a$ implies $g(a'_n) \rightarrow g(a)$, which means
$g$ is continuous at $a$. we conclude that same for $b$, using the same argument, and therefore
$g$ is continuous over $[a, b]$.

($\Leftarrow$)
$[a, b]$ is a compact set, and since $g$ is continuous over this set,
we have that $g$ is also uniformly continuous over $[a, b]$,
which means we also have that $g$ is uniformly continuous over $(a, b) \subseteq [a, b]$.
}
}