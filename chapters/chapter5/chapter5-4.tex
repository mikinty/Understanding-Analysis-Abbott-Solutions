%% 5.4 %%

\section{A Continuous Nowhere-Differentiable Function}
\setcounter{exercise}{0}

\bx{
Sketch of $(1/2)h(2x)$ on $x \in [-2, 3]$.

\begin{figure}[H]
  \centering
  \def\tickWidth{2}
  \def\scale{2}
  \begin{tikzpicture}
    \draw 
      (-2*\scale, 0) -- (3*\scale, 0)
      (-2*\scale, \tickWidth pt) -- (-2*\scale, -\tickWidth pt) node[below] {-2}
      (-1*\scale, \tickWidth pt) -- (-1*\scale, -\tickWidth pt) node[below] {-1}
      (0 *\scale, \tickWidth pt) -- (0 *\scale, -\tickWidth pt) node[below] {0}
      (1 *\scale, \tickWidth pt) -- (1 *\scale, -\tickWidth pt) node[below] {1}
      (2 *\scale, \tickWidth pt) -- (2 *\scale, -\tickWidth pt) node[below] {2}
      (3 *\scale, \tickWidth pt) -- (3 *\scale, -\tickWidth pt) node[below] {3}
      (0, 0) -- (0, 1*\scale)
      (-\tickWidth pt, 0.5*\scale) -- (\tickWidth pt, 0.5*\scale) node[right] {$1/2$}
    ;

    \draw
      (-2  *\scale, 0) -- 
      (-1.5*\scale, 0.5*\scale) -- 
      (-1  *\scale, 0) -- 
      (-0.5*\scale, 0.5*\scale) -- 
      (0   *\scale, 0) -- 
      (0.5 *\scale, 0.5*\scale) -- 
      (1   *\scale, 0) -- 
      (1.5 *\scale, 0.5*\scale) -- 
      (2   *\scale, 0) -- 
      (2.5 *\scale, 0.5*\scale) -- 
      (3   *\scale, 0)
    ;
  \end{tikzpicture}
  \caption{$(1/2)h(x)$}
  \label{fig:chap5:hx_plot}
\end{figure}
Notice that this is just the original $h(x)$ plot but $y$-scaled down by half
and the $x$-scale is squeezed towards the origin by half.

If we consider $h_n(x) = \frac{1}{2^n}h(2^nx)$ in general,
it will be $y$-scaled by $1/2^n$ and $x$-squeezed by $2^n$.
}

\bx{
We can use the fact that $h(x) \leq 1$, to show that 
\begin{equation*}
  \sum_{n=0}^\infty \frac{1}{2^n}h(2^nx)
  \leq 
  \sum_{n=0}^\infty \frac{1}{2^n}
  = 2
\end{equation*}
Since $g(x)$ is defined at every $x \in \mathbb{R}$, it is properly defined.
}

\bx{
Since we know $f(x) + g(x)$ is continuous for $f, g$ continuous,
and that $kf(x)$ is continuous if $f$ is continuous,
we see that $g_m(x)$ is a sum of constant factored continuous functions $h(x)$,
and therefore is continuous itself.

This is using the Algebraic Continuity Theorem from Chapter 4.
}

\bx{
First, it is useful to show that 
\begin{equation*}
n \in \mathbb{Z}, h(2n) = 0
\end{equation*}
We know this is true because $h(x)$ is periodic every width 2, 
and $h(0) = 0$.

\begin{align*}
  g'(0) 
  &= 
  \lim_{x_m}
  \frac{
    g(x_m) - g(0)
  }{
    x_m - 0
  }\\
  &= \frac{
    \sum_{n=0}^\infty \frac{1}{2^n}h(2^n2^{-m}) - 
    \sum_{n=0}^\infty \frac{1}{2^n}h(0)
  }{
    2^{-m}
  }\\
  &= 
  2^m\pbra{
    \sum_{n=0}^m 2^{-n}h(2^{n-m}) +
    \sum_{n=m+1}^\infty 2^{-n}h(2^{n-m})
  }\\
  &=
  2^m\pbra{
    \sum_{n=0}^m 2^{-n}2^{n-m} + \sum_{n=m+1}^\infty 2^{-n} \cdot 0
  }
  \tag{We know that $h(2n) = 0$}\\
  &= \sum_{n=0}^m 1\\
  &= m+1
\end{align*}
Since $m+1$ does not converge, we conclude that $g'(0)$ does not exist.
\label{chap5:g0_not_exist}
}

\bx{
\ea{
\item We can choose $x_m = 1 - \frac{1}{2^m}$, 
which means $x_m \rightarrow 1$
and use the same argument as Exercise
\ref{chap5:g0_not_exist}
to show that $g'(1)$ does not exist.
The only differences are that 
first, we have to argue that 
\begin{equation}
  h\pa{
    2^n \pa{1 - \frac{1}{2^m}}
  } = \begin{cases}
    1 - \frac{1}{2^m}, &n = 0\\
    2^{n-m}, &1 \leq n \leq m\\
    0, &n > m
  \end{cases}
  \label{chap5:eq:hx_cases}
\end{equation}
This is the case, because when $n \leq m$, we have that $0 \leq 2^{n-m} \leq 1$,
and we know that $h(x) = x$ when $x \in [0, 1]$.
When we have $n > m$, $2^{n-m} = 2k$, so we know $h(2k) = 0$.

Second, our denominator is $x_m - 1$, which will actually help 
us simplify this expression, see below...
\begin{align*}
  \frac{
    \sum_{n=0}^\infty \frac{1}{2^n}h(2^n x_m) - 
    \sum_{n=0}^\infty \frac{1}{2^n}h(1)
  }{
    \pa{1 - \frac{1}{2^m}} - 1
  }
  &= 
  \frac{
    \pa{
      1 - \frac{1}{2^m} +
      \sum_{n=1}^{m} 2^{-n}\cdot 2^{n-m} +
      \sum_{n=m+1}^\infty 2^{-n} \cdot 0
    }
    - 1
  }{
    -\frac{1}{2^m}
  }
  \tag{
    Using (
      \ref{chap5:eq:hx_cases}
    )
  }\\
  &=
  -2^m\pbra{
    -\frac{1}{2^m} +
    \sum_{n=1}^{m} 2^{-m} 
  }\\
  &= 1 - \sum_{n=1}^m 1\\
  &= - m + 1,
\end{align*}
which does not converge.

%% TODO: I don't think this is entirely correct, but has the basic ideas.
For $g'(1/2)$,
we can first compute that $g(1/2) = h\pa{\frac{1}{2}} + \frac{1}{2}h(1) = 1$.
Then, we choose $x_m = \frac{1}{2} - \frac{1}{2^m}$, and compute as usual,
\begin{align*}
  \frac{
    \sum_{n=0}^\infty \frac{1}{2^n}h(2^n x_m) - 
    \sum_{n=0}^\infty \frac{1}{2^n}h(1/2)
  }{
    \pa{1/2 - \frac{1}{2^m}} - 1/2
  }
  &=
  \frac{
    \sum_{n=0}^\infty \frac{1}{2^n}h(2^{n-1} - 2^{n-m}) - 
    1
  }{
    -\frac{1}{2^m}
  }\\
  &= 
  \frac{
    \pa{
      h\pa{\frac{1}{2} - \frac{1}{2^m}} +
      h\pa{1 - \frac{2}{2^m}} +
      \sum_{n=2}^{m} 2^{-n}\pa{2^{n-1} - 2^{n-m}}
    }
    - 1
  }{
    -\frac{1}{2^m}
  }\\
  &=
  -2^m\pbra{
    \frac{3}{2} - \frac{3}{2^m}
    \sum_{n=2}^{m} 2^{-m} - 1
  }\\
  &= 3\cdot 2^{m-1} -\sum_{n=0}^m -2^{m-1} + 3\\
  &= 3\cdot 2^{m-1} - m + 4.
\end{align*}
\label{chap5:gx_derivative_dne}
\item We can define for any $x = p/2^k$ the sequence 
\begin{equation*}
  x_m = \frac{p}{2^k} - \frac{1}{2^m}
\end{equation*}
and do what we've been doing in part 
\ref{chap5:gx_derivative_dne}.

%% TODO: honestly this is kinda sloppy. Not super clear what to do.
}
}

\bx{
\ea{
\item $h_m$ is not differentiable at the multiples of $\frac{1}{2^m}$,
but differentiable everywhere else.
Since $x$ is between a multiple of any power of $2$,
we conclude that $g_m$ is differentiable at $x$.

We know that 
\begin{equation*}
  \abs{
    g'_{m+1}(x) - g'_m(x)
  } = \abs{
    \sum_{i=0}^{m+1} h'_i -
    \sum_{j=0}^{m} h'_j
  } = \abs{
    h'_{m+1}
  } = 1
\end{equation*}
since the derivative of $h_k'(x)$ at anywhere defined will be 
either 1 or $-1$.
\label{chap5:gm_prime_diff}

\item The key observation for this problem is that 
\begin{equation*}
  g(x_m) = g_m(x_m) + \sum_{n=m+1}^\infty 2^{-n}h(p2^{n-m}) = g_m(x_m).
\end{equation*}
And same for $g(y_m) = g_m(y_m)$.
In addition, notice that on the interval $[x_m, y_m]$, it must be a straight line,
since for $g_m$, the granularity of the sawtooths is at $1/2^m$.
Therefore, we can conclude 
\begin{equation*}
  g'_m(x) = \frac{
    g(y_m) - g(x_m)
  }{
    y_m - x_m
  }
\end{equation*}
since $x$ sits between $x_m, y_m$, adjacent dyadic numbers.

We also need to see that $g(x) > g_m(x_m)$, since 
$g(x) = g(x_m) + \text{stuff with } h(x)$, and $h(x) \geq 0$.
\begin{figure}[H]
  \centering
  \def\coordinateWidth{4}
  \def\tickWidth{2}
  \begin{tikzpicture}
    \draw 
      (0, 0) -- (0, \coordinateWidth)
      (0, 0) -- (\coordinateWidth, 0)
    ;

    \draw 
      (0.5, \tickWidth pt) -- 
      (0.5, -\tickWidth pt) node[below] {$x_m$}
      (2, \tickWidth pt) -- 
      (2, -\tickWidth pt) node[below] {$x$}
      (3.5, \tickWidth pt) -- 
      (3.5, -\tickWidth pt) node[below] {$y_m$}
    ;

    \draw[dashed]
      (0.5, 1.5)
      node[below] {$g(x_m)$}
      node {\textbullet}
      -- 
      (3.5, 3)
      node[below] {$g(y_m)$}
      node {\textbullet}
    ;

    \draw plot [smooth, tension=1] 
      coordinates { 
        (0.5, 1.5)
        (1, 3) 
        (2,2.5)
        (2.5,3.5)
        (3.5, 3)
      }
    ;
    
    \draw 
      (1.75,3) node[above] {$g(x)$}
    ;

    \draw[cyan, line width=1pt]
      (0.5, 1.5) -- (2, 2.5)
    ;

    \draw[purple, line width=1pt]
      (3.5, 3) -- (2, 2.5)
    ;
  \end{tikzpicture}
  \caption{$g'_m(x)$ is dashed, $g(x)$ is solid.}
  \label{chap5:fig:dyadic_points}
\end{figure}
Now the rest of the problem is easy, since we can use our above graph to see geometrically,
\begin{align*}
  \frac{g(y_m) - g(x)}{y_m - x} < g'_m(x) \tag{Purple line in Figure \ref{chap5:fig:dyadic_points}}\\
  \frac{g(x_m) - g(x)}{x_m - x} > g'_m(x) \tag{Cyan line in Figure \ref{chap5:fig:dyadic_points}}
\end{align*}
\label{chap5:gm_prime_squeeze}
\item I don't think we even need part 
\ref{chap5:gm_prime_squeeze}.

We know from part
\ref{chap5:gm_prime_diff}
that $g'_m(x)$ is not a Cauchy sequence, so therefore it does not converge,
and we can conclude tht $g'(x)$ does not exist.
}
}

\bx{
We review our Cauchy Sequence argument for $g'_m(x)$, which was 
\begin{equation*}
  \abs{
    g'_{m+1}(x) - g'_{m}(x)
  } = \abs{
    h'_{m+1}(x)
  }
\end{equation*}
With $\sum_{n=0}^\infty (1/2^n)h(3^nx)$, we get that 
\begin{equation*}
  \abs{
    g'_{m+1}(x) - g'_{m}(x)
  } = \frac{1}{2^m+1} \cdot 3^{m+1} = \pa{\frac{3}{2}}^{m+1}
\end{equation*}
which means that the sequence diverges, and we can still conclude 
$g'(x)$ does not exist.

On the other hand, with $\sum_{n=0}^\infty (1/3^n)h(2^nx)$, we get that 
\begin{equation*}
  \abs{
    g'_{m+1}(x) - g'_{m}(x)
  } = \frac{1}{3^m+1} \cdot 2^{m+1} = \pa{\frac{2}{3}}^{m+1},
\end{equation*}
which actually does converge, so we cannot use this argument.
}